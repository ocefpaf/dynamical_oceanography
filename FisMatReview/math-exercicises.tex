% Style.
\documentclass[letterpaper,portuguese,12pt,pdftex]{exam}

\usepackage{setspace}
\usepackage{lineno}
% \usepackage[left=2.5cm,top=3cm,right=2.5cm]{geometry}
\usepackage[left=1.5cm,right=2.5cm,top=2cm,bottom=2cm]{geometry}

% Portuguese.
\usepackage[brazil]{babel}
\usepackage[T1]{fontenc}
\usepackage[utf8x]{inputenc}
\usepackage{textcomp}

% Font.
\usepackage{lmodern}

% Figures.
\usepackage{epsf,epsfig}

% Bibtex and extras.
\usepackage{natbib}
\usepackage{url}
\usepackage[bookmarks=false,colorlinks=true,urlcolor={green},linkcolor={green},pdfstartview={XYZ null null 1.22}]{hyperref}

% Math.
\usepackage{cancel}
\usepackage{commath}
\usepackage{mathtools}
\usepackage{amssymb,amsmath}
\everymath{\displaystyle}

\newcommand{\sen}{\operatorname{sen}}

% Exam.
\addpoints
% \printanswers  % \noprintanswers
\usepackage{color}
\definecolor{SolutionColor}{rgb}{0.8,0.9,1}
\shadedsolutions
\renewcommand{\solutiontitle}{\noindent\textbf{Solução:}\par\noindent}
\pagestyle{headandfoot}
\footer{}{Página \thepage\ de \numpages}{}
\boxedpoints
\pointsinrightmargin
\pointpoints{ponto}{pontos}
\hqword{Questão}
\hpword{Pontos}
\hsword{Nota}
% \qformat{\textbf{Question\thequestion}\quad(\thepoints)\hfill}

% PDF metadata.
\pdfinfo{% hyperref overrides this
  /Title    (Oceanografia Física Dinâmica)
  /Author   (Filipe Fernandes)
  /Creator  (Filipe Fernandes)
  /Producer (Filipe Fernandes)
  /Subject  (exercises)
  /Keywords (prova, oceanografia)
}

% Front page.
\title{Oceanografia Física Dinâmica}
\author{Prof. Filipe Fernandes}
\date{24-Feb-2014}

% Header and footer.
\pagestyle{headandfoot}
\runningheadrule
\firstpageheader{Exercícios de Matemática}{Oceanografia Dinâmica}{24 de Fevereiro, 2014}
\runningheader{Oceanografia Dinâmica}
{Exercício de Matemática, Página \thepage\ de \numpages}
{24 de Fevereiro, 2014}
\firstpagefooter{}{}{}
\runningfooter{}{}{}


\begin{document}
\maketitle
\doublespacing

\begin{center}
\fbox{\fbox{\parbox{5.5in}{
Esse Exercício incluí \numquestions\ questões.  O número total de pontos é \numpoints.

Responda as questões no espaço disponível logo abaixo.  Se o espaço
não for suficiente continue na parte de trás da folha.}}}
\end{center}
\vspace{0.1in}
\makebox[\textwidth]{Nome e número de matrícula:\enspace\hrulefill}
\vspace{0.2in}

\begin{questions}

% (1.)
\question[1]
Considere três vetores $\vec{A}$, $\vec{B}$ e $\vec{C}$.  Demonstre que:
\[
  \vec{A} \cdot \left( \vec{B} \times \vec{C} \right) = \vec{B} \cdot \left( \vec{C} \times \vec{A} \right)
\]

\makeemptybox{7cm}

% (2.)
\question
Considere a seguinte equação diferencial ordinária:
\[
  \dod[2]{}{t} - 4x = 0
\]

\begin{parts}
  \part[\half]  % (a.)
  Qual a ordem da equação?
  \fillwithlines{1cm}

  \part[\half]  % (b.)
  Encontre a solução geral.
  \makeemptybox{7cm}

  \part[1] % (c.)
  Considere que $x(t=0) = 1$ e que $x$ deve ser {\bf finito} em
  $t \rightarrow \infty$. Nesse caso, encontre a solução particular.
  \makeemptybox{7cm}
\end{parts}

% (3.)
\question
Considere a seguinte equação diferencial ordinária:
\[
  \dod[2]{}{t} + 4x = 0
\]
\begin{parts}
  \part[1]  % (a.)
  Encontre a solução geral.
  \makeemptybox{7cm}

  \part[1] % (b.)
  Considere que $x(t=0) = 1$ e que $x\left( t=\frac{\pi}{4} \right) = 1$. Nesse caso,
  encontre a solução particular.
  \makeemptybox{7cm}
\end{parts}

% (4.)
\question[1]
Defina {\bf Volume Material}.
\makeemptybox{12cm}

% (5.)
\question[2]
Expanda em série de Taylor (apenas os cinco primeiros termos) a função
$\sen{\theta}$ em torno de $\theta_o$.  Quais são os termos lineares.
\makeemptybox{12cm}

% (6.)
\question[2]
Considere a equação de Navier-Stokes em um referencial não-inercial (ou seja,
girante):
\[
  \dpd{\vec{u}}{t} + \vec{u} \cdot \nabla \vec{u} + 2\vec{\Omega} \times \vec{u} = -\dfrac{\nabla p}{\rho} + \dfrac{\mu}{\rho}\nabla^2\vec{u} + \left( g + \Omega^2R \right)\vec{k}
\]

Identifique cada termo e diga se é linear ou não-linear.

\makeemptybox{12cm}

\begin{center}
\gradetable[h][questions]
\end{center}

\end{questions}
\end{document}
