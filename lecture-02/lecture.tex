% Title page.
\title[Aula 01]{Oceanografia Física Dinâmica}
\subtitle{Conceitos de Dinâmica de Fluidos Geofísicos}
\author[Filipe Fernandes]{Filipe P. A. Fernandes}
\institute[unimonte]{Centro Universitário Monte Serrat}
\date[Fevereiro 2014]{17 de Fevereiro 2014}

\logo{\includegraphics[scale=0.15]{../common/university_logo.png}}

\begin{document}

% The title page frame.
\begin{frame}[plain]
  \titlepage
\end{frame}

\section*{Outline}
\begin{frame}
\tableofcontents
\end{frame}

\section{Conceitos de Dinâmica de Fluidos Geofísicos}
\begin{frame}
  \frametitle{O que são feições dinâmicas?}
  \begin{center}
    \shadowbox{\includegraphics[scale=0.4]{./figures/brazil_amo_2014019.png}}
  \end{center}
  \tiny\hfill \url{http://visibleearth.nasa.gov/view.php?id=82968}
\end{frame}


\begin{frame}
  \frametitle{Circulação Gerada pelo Vento}
  \begin{center}
  \movie[showcontrols=true]{\shadowbox{\phantom{\rule{8cm}{5cm}}}}
  {./figures/Perpetual_Ocean_(2005-2007)_[1080p]-xusdWPuWAoU.mp4}
  \end{center}
\end{frame}


\begin{frame}
  \frametitle{Circulação Termohalina}
  \begin{center}
  \movie[showcontrols=true]{\shadowbox{\phantom{\rule{8cm}{5cm}}}}
  {./figures/The_Thermohaline_Circulation_-_The_Great_Ocean_Conveyor_Belt-LkRQjTdTvFE.mp4}
  \end{center}
\end{frame}

\begin{frame}
\frametitle{Conservativa vs Não-conservativa}
  \begin{block}{}
    As equações que trataremos nesse curso possuem, em geral, duas formas:
    "forma conservativa" e "forma não conservativa".
  \end{block}

  \begin{block}{}
    A forma conservativa enfatiza a interpretação física das equações como leis
    de conservação por um volume fixo no espaço.

    A forma não conservativa enfatiza mudanças no estado do volume de controle
    a medida que ele move com o fluido.
  \end{block}
\end{frame}

\begin{frame}
\frametitle{Leis de Newton}
  \begin{itemize}[<+-| alert@+>]
    \item Primeira Lei: Quando analisado sobre um referencial inercial, um
    objeto está ou em repouso ou em velocidade constante a menos que uma força
    aja sobre ele.

    \item Segunda Lei: A aceleração de um objeto é diretamente proporcional a,
    e na mesma direção, que a soma vetorial das forças que agem nesse objeto.
    $F = ma$.

    \item Terceira Lei: Para toda ação tem uma reação de mesma magnitude na
    direção oposta.
  \end{itemize}
\end{frame}

\subsection{Equação de Euler}
\begin{frame}
\frametitle{Equação de Euler}
  \begin{block}{}
    A equação de {\bf Euler}, em Dinâmica de Fluidos Geofísicos, é um conjunto de
    equações  que governos fluidos invíscidos.
  \end{block}

  \begin{block}{}
    As equações representam a conservação de massa (ou continuidade), momento
    e energia.  Corresponde as equações de {\bf Navier–Stokes} com os termos de
    viscosidade e condução de calor zero.
  \end{block}
\end{frame}

\begin{frame}
\frametitle{Equação de Euler}
\[
    {\partial \rho  \over \partial t}+\nabla \cdot (\rho {\mathbf  u})=0
\]
Onde:
    \begin{itemize}
      \item $\rho$ é a massa do fluído;
      \item ${\mathbf  u}$ é o vetor velocidade (com as componentes $u$, $v$
            e $w$).
    \end{itemize}
\end{frame}


\begin{frame}
\frametitle{Equação de Navier–Stokes}
\[
\rho \Big(
\underbrace{\frac{\partial \mathbf{u}}{\partial t}}_{
\begin{smallmatrix}
  \text{Aceleração}\\
  \text{local}
\end{smallmatrix}} +
\underbrace{\mathbf{u} \cdot \nabla \mathbf{u}}_{
\begin{smallmatrix}
  \text{Aceleração} \\
  \text{Advectiva}
\end{smallmatrix}}\Big) =
\underbrace{-\nabla p}_{
\begin{smallmatrix}
  \text{Gradiente de} \\
  \text{pressão}
\end{smallmatrix}} +
\underbrace{\mu \nabla^2 \mathbf{v}}_{\text{Viscosidade}} +
\underbrace{\mathbf{f}}_{
\begin{smallmatrix}
  \text{Forças} \\
  \text{de} \\
  \text{corpo}
\end{smallmatrix}}
\]
Nota: A equação acima é válida apenas para fluidos Newtonianos
não-compressíveis.
\end{frame}

\begin{frame}
\frametitle{Equações do movimento (juntando tudo)}
  \begin{itemize}[<+-| alert@+>]
    \item vetorial: $\frac{D{\mathbf{u}}}{Dt} = -\frac{1}{\rho}\nabla{p} -2\mathbf{\Omega} \times \mathbf{u} + \mathbf{g}\hat{k} + \mathbf{Fr}$
    \item x: $\frac{Du}{Dt} = fv - \frac{1}{\rho}\pd{p}{x} + \left(\pd{\tau^{xx}}{x} + \pd{\tau^{xy}}{y} + \pd{\tau^{xz}}{z}\right)$
    \item y: $\frac{Dv}{Dt} = -fu - \frac{1}{\rho}\pd{p}{y} + \left(\pd{\tau^{xy}}{x} + \pd{\tau^{yy}}{y} + \pd{\tau^{yz}}{z}\right)$
    \item z: $\frac{Dw}{Dt} = - \frac{1}{\rho}\pd{p}{z} - g + \left(\pd{\tau^{xz}}{x} + \pd{\tau^{yz}}{y} + \pd{\tau^{zz}}{z}\right)$
    \item $\pd{u}{x} + \pd{v}{y} + \pd{w}{z} = 0$
  \end{itemize}
\end{frame}


\begin{frame}
\frametitle{Dever de -- casa 01}
    \begin{block}{}
      \href{http://en.wikipedia.org/wiki/Euler_equations_(fluid_dynamics)}{Euler: \url{http://en.wikipedia.org/wiki/Euler_equations_(fluid_dynamics)}}
    \end{block}

    \begin{block}{}
      \href{http://en.wikipedia.org/wiki/Navier-Stokes_equations}{Navier–Stokes: \url{http://en.wikipedia.org/wiki/Navier-Stokes_equations}}
    \end{block}
    {\scriptsize Note as formas cilíndricas e esféricas das mesmas equações.}
    {\scriptsize * Digite no YouTube "Non-newtonian fluid"}
\end{frame}

\section{Coriolis}
\begin{frame}
  \frametitle{Quem tem medo da rotação da Terra desce agora!}
  \begin{center}
    \includegraphics[scale=0.6]{./figures/calvin_rotation.png}
  \end{center}
\end{frame}


\begin{frame}
  \frametitle{Mas será que ela gira mesmo?}
  \begin{center}
    \includegraphics[scale=0.4]{./figures/galileo_goes_to_jail.jpg}
  \end{center}
\end{frame}


\begin{frame}
  \frametitle{Trilha estelar 01}
  \begin{center}
    \includegraphics[scale=0.45]{./figures/NCPtreeLosada.jpg}
%     http://apod.nasa.gov/apod/ap131023.html
  \end{center}
% North Celestial Pole at the center of all the star trail arcs recorded over a
% period of nearly 2 hours as a series of 30 second long, consecutive exposures
% on the night of October 5 near Almaden de la Plata, province of Seville, in
% southern Spain. The axis of rotation leads to the center of the concentric
% arcs in the night sky. For northern hemisphere the bright star Polaris is very
% close to the North Celestial Pole and so makes the short bright trail in the
% central gap between the leafy branches.
\end{frame}


\begin{frame}
  \frametitle{Trilha estelar 02}
  \begin{center}
    \includegraphics[scale=0.45]{./figures/2007_09_14-orion-lq_vangorp1200.jpg}
  \end{center}
% Made on September 14 from Montlaux, France, this wide-angle view nicely shows
% the stars near the celestial equator tracing nearly straight lines in
% projection, while stars north and south of the equator, respectively, appear
% to circle the north and south celestial poles. Featured are the stars of Orion
% (right of center), brilliant Venus rising (left) as bright star Sirius rises
% in the south (bottom center), and a polar orbiting Iridium satellite (upper
% left).  This picture was constructed from 477 consecutive 30 second digital
% exposures recorded over 4.3 hours.
\end{frame}


\begin{frame}
  \frametitle{Trilha estelar 03}
  \begin{center}
%     http://apod.nasa.gov/apod/ap040708.html
    \includegraphics[scale=4]{./figures/040523cruxa_seip_full.jpg}
% Gradually change the focus of the camera lens during the exposure, and you
% could end up with a dramatic picture like this one where the out-of-focus
% portion of the trail shows off the star's color.  Crux, the Southern Cross.
% Gacrux or gamma Crucis is the bright red giant star only 88 light-years
% distant that forms the top of the Cross seen here near top center. Acrux, the
% hot blue star at the bottom of the Cross is about 320 light-years distant.
% Actually a binary star system, Acrux is the alpha star of the compact Southern
% Cross and lies along a line pointing from Gacrux to the South Celestial Pole,
% off the lower right edge of the picture. Adding a separate short exposure to
% the end of the step-focussed trails to better show the positions of the stars
% themselves. The dark night skies above Namibia.
  \end{center}
\end{frame}


\begin{frame}
  \frametitle{Trilha estelar 04}
  \begin{center}
  \movie[showcontrols=true]{\centerline{\includegraphics[scale=0.7]{./figures/hawaiian_starlight_rotatingstarfield-apod.png}}}
  {./figures/hawaiian_starlight_rotatingstarfield-apod.flv}
  \end{center}
\end{frame}

\begin{frame}
  \frametitle{A Terra não gira apenas no próprio eixo}
  \begin{center}
    \includegraphics[scale=0.35]{./figures/image_0303_analemma0600ut_ayiomamitis_full.jpg}
  \end{center}
\end{frame}


\begin{frame}
  \frametitle{E o eixo não está onde gostaríamos...}
  \begin{center}
    \includegraphics[scale=0.7]{./figures/AxialTiltObliquity.png}
%     http://upload.wikimedia.org/wikipedia/commons/6/61/AxialTiltObliquity.png
  \end{center}
\end{frame}


\begin{frame}
  \frametitle{Formulando tudo isso}
  \begin{block}{Velocidades}
    \[
      \vec{u_{\mathbf{f}}} = \vec{u_{\mathbf{r}}} + \vec{\Omega} \times \vec{r}
    \]
  \end{block}

  \begin{block}{Acelerações}
    \[
      \vec{a_{\mathbf{f}}} = \vec{a_{\mathbf{r}}} + 2 \vec{\Omega} \times
      \vec{u_{\mathbf{r}}} - \Omega^2\vec{R}
    \]
  \end{block}
\end{frame}


\begin{frame}
  \frametitle{Só nos interessa a componente vertical}
  \begin{columns}
    \begin{column}{0.5\textwidth}
      \begin{itemize}
        \item $\Omega_x = 0$
        \item $\Omega_y = \Omega\cos\theta$
        \item $\Omega_z = \Omega\sin\theta$
      \end{itemize}
    \end{column}
    \begin{column}{0.5\textwidth}
      \begin{center}
        \includegraphics[scale=0.4]{./figures/coriolis_vertical.png}
      \end{center}
    \end{column}
  \end{columns}
\end{frame}

\begin{frame}
  \frametitle{De onde chegamos a $f = 2\Omega\sin{\theta}$?}
  \[
    2\vec{\Omega} \times \vec{u} =
    \begin{vmatrix}
    \vec{i} & \vec{j}           & \vec{k}\\
    0       & 2\Omega\cos\theta & 2\Omega\sin\theta\\
    u       & v                 & w
    \end{vmatrix}
  \]
  \pause
  \[
    2\vec{\Omega} \times \vec{u} =
    2\Omega [\xcancel{\vec{i}(w\cos\theta} - v\sin\theta) +
             \vec{j}u\sin\theta - \vec{k}u\cos\theta]
  \]
  \pause
  \begin{itemize}
    \item $(2\vec{\Omega} \times \vec{u})_x = -(2\Omega\sin\theta)v = -fv$
    \item $(2\vec{\Omega} \times \vec{u})_y = (2\Omega\sin\theta)u = fu$
    \item $(2\vec{\Omega} \times \vec{u})_z = \cancelto{*}{-(2\Omega\cos\theta)u}$
  \end{itemize}
\end{frame}


\begin{frame}
  \frametitle{23 horas, 56 minutos, 4.0916 segundos}
  \begin{itemize}[<+-| alert@+>]
    \item $\Omega = \dfrac{2\pi} {24 \times 60 \times 60} =
           7,272 \times 10^{-5}$ rad s$^{-1}$
    \item $\Omega^* = \dfrac{2\pi + 2\pi/360} {24 \times 60 \times 60} =
           7,292 \times 10^{-5}$ rad s$^{-1}$
  \end{itemize}
  \pause
  \small{* Esse é o $\Omega$ que devemos usar incluindo a rotação da Terra ao
         redor do Sol.  Existem outros efeitos menores como o Movimento do
         Sistema Solar na Galáxia e o Movimento de Galáxia no Universo.}
\end{frame}

\begin{frame}
  \frametitle{Alguns números}
  \begin{itemize}[<+-| alert@+>]
    \item $\Omega = 7,292 \times 10^{-5}$ rad s$^{-1}$
    \item Máxima aceleração de Coriolis (nos polos):

        $2\Omega \text{v} \sim 1,5 \times 10^{-4}\text{v}$

        v = 10000 m s$^{-1}  \rightarrow 1,5$ m s$^{-1}$
    \item $\dfrac{d\Omega}{dt} = 0$
  \end{itemize}
\end{frame}


\begin{frame}
  \frametitle{Onde foi parar $\Omega^2\vec{R}$?}

  \begin{columns}
    \begin{column}{0.5\textwidth}
    \small{
      Aceleração centrípeta no equador:
      $\Omega^2\text{R}_{\text{e}}^{*} = 0,0338$ m s$^{-2}$

      R$_{\text{equador}} -$ R$_{\text{polo}} \approx 21,4 \times 10^{3}$ m

      g$_{\text{equador}} -$ g$_{\text{polo}} \approx -0,052$ m s$^{-2}$
      }
    \end{column}
    \begin{column}{0.5\textwidth}
      \begin{center}
        \includegraphics[scale=0.6]{./figures/gravity_centrigugal_vectors.png}
      \end{center}
    \end{column}
  \end{columns}
\end{frame}


\begin{frame}
  \frametitle{Dever de casa -- 02}
  \begin{center}
    \includegraphics[scale=0.55]{./figures/ocean_description.png}
  \end{center}

\scriptsize{

\href{http://oceanmotion.org/html/resources/coriolis.htm}{Circulação Inercial: \url{http://oceanmotion.org/html/resources/coriolis.htm}}

\href{http://www.britishpathe.com/video/gambling-with-gulf-stream-aka-gambling-on-the-gulf}{Corrente do Golfo: \url{http://www.britishpathe.com/video/gambling-with-gulf-stream-aka-gambling-on-the-gulf}}

\href{http://esminfo.prenhall.com/science/geoanimations/animations/26_NinoNina.html}{ENSO: \url{http://esminfo.prenhall.com/science/geoanimations/animations/26_NinoNina.html}}

}

* Obs: Há também uma lista com exercícios práticos no portal!
\end{frame}

\end{document}
