% Style.
\documentclass[letterpaper,portuguese,12pt,pdftex]{exam}

\usepackage{setspace}
\usepackage{lineno}
\usepackage[left=2.5cm,top=3cm,right=2.5cm]{geometry}

% Portuguese.
\usepackage[brazil]{babel}
\usepackage[T1]{fontenc}
\usepackage[utf8x]{inputenc}
\usepackage{textcomp}
\usepackage{commath}

% Font.
\usepackage{lmodern}

% Figures.
\usepackage{epsf,epsfig}

% Bibtex and extras.
\usepackage{natbib}
\usepackage{url}
\usepackage[bookmarks=false,colorlinks=true,urlcolor={green},linkcolor={green},pdfstartview={XYZ null null 1.22}]{hyperref}

% Math.
\usepackage{amssymb,amsmath}
\usepackage{mathtools}
\usepackage[makeroom]{cancel}
\everymath{\displaystyle}

% Exam.
\addpoints
% \printanswers
\usepackage{color}
\definecolor{SolutionColor}{rgb}{0.8,0.9,1}
\shadedsolutions
\renewcommand{\solutiontitle}{\noindent\textbf{Solução:}\par\noindent}
\pagestyle{headandfoot}
\footer{}{Página \thepage\ de \numpages}{}
\boxedpoints
\pointsinrightmargin
\pointpoints{ponto}{pontos}
\hqword{Questão}
\hpword{Pontos}
\hsword{Nota}
% \qformat{\textbf{Question\thequestion}\quad(\thepoints)\hfill}

% PDF metadata.
\pdfinfo{% hyperref overrides this
  /Title    (Prova 01 -- Oceanografia Física Dinâmica)
  /Author   (Filipe Fernandes)
  /Creator  (Filipe Fernandes)
  /Producer (Filipe Fernandes)
  /Subject  (prova)
  /Keywords (prova, oceanografia)
}

% Front page.
\title{Prova 01 -- Oceanografia Física Dinâmica}
\author{Prof. Filipe Fernandes}
\date{14-Abril-2014}

\begin{document}
\maketitle
\doublespacing

\hbox to \textwidth{Nome e número de matrícula:\enspace\hrulefill}
\vspace{0.5cm}

\begin{minipage}{.8\textwidth} % 35
Esse exame incluí \numquestions\ questões. O número total de pontos é \numpoints.
\vspace{1cm}
{\small
\begin{itemize}
  \item Coloque seu nome em todas as folhas e numere as mesma colocando o
            número total de folhas. (Ex.: 1/4, 2/4, 3/4 e 4/4.)
  \item Coloque suas respostas apenas na folha de respostas (pode usar
            as folhas da prova como rascunho).
  \item Essa prova segue o Acordo Ortográfico da Língua Portuguesa de 1990 (em
            vigor no início de 2009).  Por isso erros de ortografia e gramática
            serão descontados da sua nota final.
  \item A prova deve ser feita individualmente e sem consulta.
  \item Use caneta (preta ou azul) para responder as questões – qualquer questão
            respondida à lápis não será considerada na hora da correção.
  \item Leia atentamente todas as questões: a interpretação faz parte da prova.
            Dúvidas serão esclarecidas apenas após o término da mesma.
  \item Desligue e guarde o celular.  Celulares à vista serão recolhidos!
\end{itemize}
}
\end{minipage}

\newpage

\begin{questions}
  \question
  Responda as questões abaixo utilizando as diversas ``versões'' e/ou
  componentes da equação do movimento (ou conservação de momento) fornecidas
  nos itens (a), (b) e (c).
  \begin{align*}
  \text{(a) } \dod{w}{t} &= -\dfrac{1}{\rho}\dfrac{\partial p}{\partial z} - g + \nu\nabla^2w\\
  \text{(b) } \dfrac{\Dif \mathbf{u}}{\Dif t} &= -\dfrac{1}{\rho}\nabla{p} - 2\mathbf{\Omega} \times \mathbf{u} + \mathbf{g} + \mathbf{\mathbf{F_R}}\\
  \text{(c) } \dpd{u}{t} + u\dpd{u}{x} + u\dpd{v}{y} + u\dpd{w}{z} &= -\dfrac{1}{\rho}\dpd{p}{x} -fv + \text{A}_{\text{h}}\left( \dpd[2]{u}{x} + \dpd[2]{u}{y} + \dpd[2]{u}{z} \right)
  \end{align*}

  \begin{parts}
  \part[3]
  Abra o termo $\dfrac{\Dif \mathbf{u}}{\Dif t}$ e explique todos os sub-termos
  que você escrever.
  \begin{solution}
  \begin{alignat*}{4}
    & \dpd{u}{t} &&+ u\dpd{u}{x} &&+ u\dpd{v}{y} &&+ u\dpd{w}{z} \\
    & \dpd{v}{t} &&+ v\dpd{u}{x} &&+ v\dpd{v}{y} &&+ v\dpd{w}{z} \\
    & \dpd{w}{t} &&+ w\dpd{u}{x} &&+ w\dpd{v}{y} &&+ w\dpd{w}{z}
  \end{alignat*}
  $\dpd{}{t} \rightarrow$ Ac. local., e $\mathbf{u}\nabla\mathbf{u}$ ac.
  advectiva.
  \end{solution}


  \part[3]
  Identifique qual(is) componente(s) do sistema de coordenada cartesiano está
  representado em cada equação.
  \begin{solution}
    \begin{itemize}
      \item[(a)] Componente vertical ($z$).
      \item[(b)] Forma vetorial ($x$, $y$ e $z$).
      \item[(c)] Componente horizontal ($x$).
    \end{itemize}
  \end{solution}

  \part[3]
  Identifique e explique os termos de aceleração dos itens (a), (b) e (c).
  \begin{solution}
    \begin{itemize}
      \item[(a)] Aceleração total na vertical.
      \item[(b)] Aceleração total do fluido.
      \item[(c)] Aceleração local + advectiva na horizontal ($x$).
    \end{itemize}
  \end{solution}


    \part[3]
    Encontre os termos de atrito de (a) e (c) e explique as diferenças e
    semelhanças entre eles.
    \begin{solution}
    O item (a) usa o coeficiente de viscosidade cinemática $\nu$ e a forma
    compacta $\nabla^2$ (laplaciano) como operador em $w$, já o item (b)
    ``abre'' o operador laplaciano e utiliza o coeficiente de viscosidade
    turbulenta na horizontal A$_\text{h}$.
    \end{solution}

  \part[3]
  Cancele os termos não-lineares de (c), faça a aceleração local igual a zero
  (sistema não acelerado), e explique o balanço encontrado.
  \begin{solution}
  \begin{align*}
  \cancelto{0}{\dpd{u}{t}} + \cancelto{\text{não-linear}}{\dpd{u}{x} + u\dpd{v}{y} + u\dpd{w}{z}} &= -\dfrac{1}{\rho}\dpd{p}{x} -fv + \cancelto{\text{não=linear}}{\text{A}_{\text{h}}\left( \dpd[2]{u}{x} + \dpd[2]{u}{y} + \dpd[2]{u}{z} \right)} \\
  0 &= -\dfrac{1}{\rho}\dpd{p}{x} -fv \\
  fv &= -\dfrac{1}{f\rho}\dpd{p}{x}
  \end{align*}
  Coriolis e Força Gradiente = Pressão (balanço geostrófico).
  \end{solution}

  \end{parts}

  \question[5]
  Imagine que o seu {\it Jaeger} recebeu a missão de caçar um {\it Kaiju} nas
  fossa das Marianas, borda do Pacífico próximo a Guam.  Faça o cálculo da
  pressão hidrostática para convencer o seu comandante que o seu robô gigante
  não vai aguentar a pressão!

  Considere um estado não-acelerado na coordenada vertical $z$ e atrito
  total do fluido como praticamente nulo.  (Para seus cálculos use gravidade de
  9,8 m s$^{-2}$, densidade média de 1025 kg m$^{-3}$ e profundidade de 11 km.)

  \begin{solution}
    \begin{align*}
      \cancelto{0}{\dfrac{dw}{dt}} &= -\dfrac{1}{\rho}\dfrac{\partial p}{\partial z} - g + \cancelto{0}{\nu\nabla^2w}\\
      \dfrac{1}{\rho}\dfrac{\partial p}{\partial z} &=  -g \\
      \partial p &= -g\rho\partial z \\
      \int^{-z}_{0}\partial p &= -g\rho\int^{-z}_{0}\partial z \\
      p(-z) -\cancelto{0}{p(0)} &= -g\rho(-z-0) \\
      p(-z) &= \rho g z \\
      p(-11000) &= 1025 \times 9.8 \times 11000 \\
      p(-11000) &= 110495000.0 \text{ Pa ou } 11049,5 \text{ dbar}
    \end{align*}
  \end{solution}

  \question[5]
  Explique a frase: ``Coriolis não é realmente uma força em si, apenas uma
  questão de como você olha para ela.''

  \begin{solution}
    Coriolis é uma força fictícia que usamos para ``corrigir'' um sistema de
    coordena não-inercial, ou seja, que move juntamente com o sistema de
    coordenada usados.
  \end{solution}

  \question[2\half]
  Conhecemos a equação: \( \dpd{u}{x} + \dpd{v}{y} + \dpd{w}{z} = 0\) como
  equação da continuidade.  Mas vimos em sala de aula que ela também representa
  uma conservação similar à conservação de sal e calor.  Explique o que ela
  representa e dê um exemplo de sua aplicação.

  \begin{solution}
    Representa a conservação de massa para oceanos homogêneos ($\rho =$
    constante).  Qualquer explicação como ressurgência e/ou Stommel \& Arons
    está OK.
  \end{solution}


  \begin{minipage}[b]{0.85\linewidth}
    \question[2\half]
    A figura ao lado mostra um esquema dos tensores (componentes do atrito) em
    um elemento infinitesimal de fluido.  Imagine os seguintes cenários: (a) um
    vento aplicado apenas na superfície do oceano; (b) atrito interno nas
    camadas do fluido, e (c) atrito com o fundo.  Qual(is) $\tau_{ij}$ você
    usaria nos cenários (a), (b) e (c)?
  \end{minipage}
  \hfill
  \begin{minipage}[b]{0.35\linewidth}
    \includegraphics[height=5\baselineskip]{./figures/stresses.png}
  \end{minipage}

  \begin{solution}
    \begin{itemize}
      \item[(a)] $\tau_{zx}$ e $\tau_{zy}$.
      \item[(b)] Todos!
      \item[(c)] Novamente $\tau_{zx}$ e $\tau_{zy}$.
    \end{itemize}

  \end{solution}


  \question[5]
  Comente sobre cada uma das forças abaixo.  Diga se é: de superfície
  (ou contato) ou de corpo (volume), fictícia ou ``real'', primária (que causa)
  movimento ou secundária (oriunda do movimente). (Dica: organize a sua resposta
  em uma tabela.)

  {\bf Forças}: Gradiente de Pressão, Coriolis, Atrito, Gravidade e Centrífuga.
  \begin{solution}
    \begin{tabular}{l|l|l|l}
        & Contato/Volume & Real/Fictícia & Primária/Secundária \\
    \hline
    FGP & Volume         & Real          & Primária \\
    FC  & Volume         & Fictícia      & Secundária \\
    FA  & Contato        & Real          & Secundária ou Primária* \\
    FG  & Volume         & Real          & Primária \\
    FCF & Volume         & Fictícia      & Secundária \\
    \hline
    \end{tabular}

    * Quando o atrito é entre as camadas de água ou com o fundo ela é uma força
    secundária, mas quando é com o vento ela é primária.
  \end{solution}

  \bonusquestion[5]
  Pergunta bônus.

  Dê pelo menos um exemplo para fenômenos de pequena, meso e larga escala.
  Discorra sobre a sua escolha explicando como ele se encaixa na classificação
  acima. ({\bf Atenção}!!! A pergunta bônus será considerada somente se a
  resposta estiver 100\% correta!)

\end{questions}

\end{document}
